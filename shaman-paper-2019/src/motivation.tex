\section{Motivation and significance}
\label{motivation}

Development efforts for UI libraries and frameworks dedicated to desktop and workstation software
have virtually stalled compared with modern web UI frameworks. User expectations for most software
packages center on one experience: is the software available as a reactive web or mobile
application? The interest in such a deployment mechanism comes from the convenience of immediate use
and the lack of any installation time or maintenance. 

The same preferences can be found with users of neutron scattering facilities—such as HFIR and
SNS—who have significantly compressed schedules during their visits. The demands of their
experiments leave little time for complicated software and installation procedures. Furthermore,
when these users return to their institutions, debugging remote issues on native desktop software
can be extremely time consuming, if not impossible, for the development team. In contrast to their
native counterparts, web UIs have no installation time, and problems are relatively easy to diagnose
remotely. Web UIs also have the added benefit of central deployment in which fixing a bug for one
user immediately fixes the bug for all users (to within cache refresh times).

Shaman is a new web interface for data reduction for the Bio-SANS, EQ-SANS, and GP-SANS instruments
at HFIR and SNS. Fundamental goals of its development include quick turnaround for users and easy
access to data. This includes streamlining the workflow to quickly and reactively prepare the
reduction configuration needed to reduce the data, as well as integrated interactive visualization
and data export tools for working with that data. The user experience across all three instruments
is the same with identification of the specific instrument required only for gathering the correct
data and launching the correct reduction program on the server. 