Introduction and Organization

This report is a status update on the progress of software development for data reduction software
for the small-angle neutron scattering (SANS) instruments at Oak Ridge National Laboratory’s
(ORNL’s) High Flux Isotope Reactor (HFIR) and Spallation Neutron Source (SNS). This software was
developed in response to findings from the 2018 triennial review of HFIR and SNS and as part of the
Software Development Plan for SNS and HFIR (“the plan”) that was reviewed in July 2019 and approved
in October 2019. This document is provided to answer the Department of Energy Basic Energy Sciences’
request to provide a summary status report by December 31, 2019.

The following sections provide information on the two primary parts of the SANS reduction software:
the Shaman web interface and the Data Reduction Toolkit for SANS (drt-sans). Both packages were
developed with feedback from users and instrument scientists. The instrument scientists joined the
development team as product owners in the software development process. In this role, the instrument
scientists provided tests and, in some cases, wrote test code for drt-sans. Tests and reduction code
were validated by members of the instrument team once implemented in drt-sans. The instrument
scientists also provided detailed design guidelines for the layout and contents of the configuration
tools provided in Shaman. Specific design choices for Shaman and drt-sans were made based on user
feedback gathered in May and June 2019, including feedback gathered during a lunch session at the
annual SNS-HFIR User Group meeting. Drt-sans was used during instrument commissioning for the
General-Purpose SANS Diffractometer (GP-SANS) and Biological SANS Instrument (Bio-SANS) when HFIR
returned to operations in November 2019. Shaman has since been used with data gathered during
commissioning.

Shaman and drt-sans were developed jointly between the Scientific Computing and Software Engineering
(SCSE) group in the Neutron Scattering Division and the Research Software Engineering (RSE) group in
the Computer Science and Mathematics Division. These two groups are matrixed together through a
common group leader (Billings) and share staff, computational resources, and a common development
process. Successful completion of the development effort was greatly facilitated by this
organizational structure. The development team uses an Agile, test-driven development process that
requires strong engagement with stakeholders and users to deliver state-of-the-art solutions.

The first section describes the new web interface, Shaman, that provides a uniform and easy-access
graphical user interface (UI) that users can access from their normal web browsers to reduce data
for the Bio-SANS, Extended Q-Range SANS (EQ-SANS), and GP-SANS instruments. Shaman does not require
any programming skills or detailed knowledge of the reduction software it executes, and it does not
require users to install software. It greatly simplifies the reduction workflow for users by
integrating the reduction software into a single tool that functions for all three SANS instruments
and uses existing, common mechanisms for data archiving and cataloging.

The second section describes drt-sans, which includes all necessary code for reducing raw, measured
SANS data to quantities—such as the scattering intensity I(q)—that are ready for analysis and
interpretation. Drt-sans is a command line program that takes a simple input file containing all the
basic information about an experiment and desired operations in the reduction process and produces
fully reduced data. The output of drt-sans includes numerical data files, images, and interactive
plots that are saved in a shared file system and read by Shaman for users to inspect. This package
relies heavily on the MANTID data reduction framework for neutron scattering but was designed such
that Shaman can issue data reduction jobs remotely on the ORNL neutrons analysis cluster.

Shaman: The Web Interface for SANS Data Reduction

Motivation and Significance

Development efforts for UI libraries and frameworks dedicated to desktop and workstation software
have virtually stalled compared with modern web UI frameworks. User expectations for most software
packages center on one experience: is the software available as a reactive web or mobile
application? The interest in such a deployment mechanism comes from the convenience of immediate use
and the lack of any installation time or maintenance. 

The same preferences can be found with users of neutron scattering facilities—such as HFIR and
SNS—who have significantly compressed schedules during their visits. The demands of their
experiments leave little time for complicated software and installation procedures. Furthermore,
when these users return to their institutions, debugging remote issues on native desktop software
can be extremely time consuming, if not impossible, for the development team. In contrast to their
native counterparts, web UIs have no installation time, and problems are relatively easy to diagnose
remotely. Web UIs also have the added benefit of central deployment in which fixing a bug for one
user immediately fixes the bug for all users (to within cache refresh times).

Shaman is a new web interface for data reduction for the Bio-SANS, EQ-SANS, and GP-SANS instruments
at HFIR and SNS. Fundamental goals of its development include quick turnaround for users and easy
access to data. This includes streamlining the workflow to quickly and reactively prepare the
reduction configuration needed to reduce the data, as well as integrated interactive visualization
and data export tools for working with that data. The user experience across all three instruments
is the same with identification of the specific instrument required only for gathering the correct
data and launching the correct reduction program on the server. 

Software Description

The primary workflow of Shaman begins with user authentication and instrument selection. Once the
instrument is selected, users are presented with the list of experiments associated with their
account in a proposal tracking system. Experiment selection is followed by parameter configuration
and the selection of individual runs from the experiment that will be used as part of the reduction.
Each run is further distinguished by the order in which it should be reduced and whether or not the
output of reducing that run should be combined with its peers through stitching, such as would be
done for samples measured with multiple configurations of an instrument to obtain a larger
measurement range. When all data reduction runs are appropriately configured, the data are reduced
by calling the reduction code on the remote analysis server. Users are provided with live feedback
on the status of the reduction workflow. The reduced data for all runs are available for user
inspection after the reduction is complete. 

Shaman has three deployments on ORNL’s servers, including development, quality assurance testing,
and production installations for developers, testers, and users, respectively. Each instance is
authenticated against ORNL’s eXternal Centralized Access Management System (XCAMS) so that users can
identify themselves with the same credentials they use for the other systems at the facilities. The
installations are hosted across various servers at ORNL depending on needs for availability and
uptime requirements. For example, the development installation is hosted on a machine with a very
flexible configuration. The production installation, by comparison, is hosted on a highly stable,
virtualized, and load-balanced cluster in ORNL’s private cloud, the Compute Advanced Data
Environment for Science (CADES), and is designed to handle far greater amounts of traffic than
either the development or testing servers. Installation of software for each deployment is done
automatically without developer involvement using continuous integration/continuous deployment tools
and techniques. The Shaman servers are separate from the servers on which the data reduction code is
executed by design. Shaman has its own dedicated computational resources that do not detract from
the considerable amount of memory and computational resources required by the reduction code, which
is run on ORNL’s neutron data analysis cluster. Shaman executes the reduction code using a remote
command framework from the Eclipse Integrated Computational Environment, which was also developed at
ORNL as a tool for managing interactive workflows.

Shaman was developed in the Java programming language, version 11.0, using the Vaadin framework,
version 14.0. Vaadin is a framework for making responsive web UIs. The core strength of Vaadin is
its ability to “transcode” normal Java code compiled on the web server into client-side JavaScript
that is served to users at the page request time, which makes it possible to use an enterprise
language such as Java for both the UI business logic and the code that draws the UI on the screen.
Shaman also uses the Spring framework for tasks not specifically related to drawing views. Spring is
a framework for building Java web services and provides several important functions to Shaman,
including authentication and service resolution. Compilation is handled with Maven, a popular Java
build system. Once compiled, the application is executed in the same way as any other Java-based web
server.

Visualization in Shaman is provided for users to inspect their raw and reduced data. Users can
access available plots statically or interactively for closer inspection of fine features.
Visualization is performed using the mpld3 library coupled to Vaadin. The mpld3 library creates
plots using Matplotlib and converts those plots to the d3 JavaScript visualization library.

All data is available for users to download to their own machine, including raw data, configuration
files, reduced data, static images, and files used by the interactive plotting engine. Users can
simply click an onscreen button to download an archive file in the popular Zip format that contains
all of this information.

Future of the Shaman Platform

The Shaman platform is a promising step in the right direction for providing users with easy-to-use
web tools for reducing data at HFIR and SNS. Shaman was designed with multiple instruments in mind
and was implemented with standard languages and tools for creating advanced web platforms.
Therefore, much of its code can be reused for instruments that are outside of the SANS instrument
suite with only a very minor and reasonable amount of refactoring. The software is also highly
maintainable because it was developed using community frameworks and does not rely on large amounts
of custom code for common services.

The diffraction instruments at HFIR and SNS are expected to be the first suite of instruments
outside of SANS for which the code in Shaman will be reused. Furthermore, reusing Shaman for this
purpose is expected to not only provide a desirable user experience with advanced tooling but also
to be highly efficient because some amount of UI code is common across all instrument techniques.
Data Reduction Toolkit for SANS

Motivation and Significance

The drt-sans effort sprung from a realization that the three SANS instruments at HFIR and SNS were
using separate tools for data reduction, which is also true at other neutron scattering facilities.
However, there are significant overlaps in the instrument design and physics of these instruments.
Bio-SANS and GP-SANS are nearly identical, with the largest design difference being that GP-SANS has
a longer detector tank and Bio-SANS has an additional wing detector. EQ-SANS, although a
time-of-flight instrument instead of a monochromatic SANS, shares many physical features with the
GP-SANS and Bio-SANS instruments and uses the same underlying data reduction concepts as the other
instruments. Thus, there is no need for three separate frameworks for data reduction, and much
efficiency can be gained from a single, unified framework.

Another important aspect of the development of drt-sans is that it does not strictly require a
graphical UI for users to execute it (i.e., it is headless). The significance of this design choice
is that it allows for computational resources to be allocated specifically and separately for
reduction performed with drt-sans and graphical UI activities performed by other tools. Previous
reduction tools for SANS at SNS and HFIR were plagued by issues commonly found when mathematical
routines are connected to outdated or large graphical tool suites. Common errors and bugs, such as
UI thread locks while waiting for work to complete, simply do not exist for drt-sans and its
companion web interface, Shaman.

Software Description

The reduction workflow in drt-sans across the three instruments is largely the same, having only
minor differences for instrument-specific features. Bio-SANS is an example of this because it has
the extra wing detector that must be dealt with in addition to the primary detector. Broadly, the
data reduction process executes as follows. First, all input parameters and physical configuration
details are loaded and rectified. Next, the data are loaded and the beam center computed. The
reduction process then begins with special preparations and corrections applied to the data
including transmission correction, background subtraction, and normalization. The data are then
converted separately to scalar and azimuthal forms, and the scattering intensity I(q) is computed
for each form. 

The drt-sans software is written in Python, which is a flexible programming language well suited to
scientific problems, and which is also easily modified when things change quickly. It is built on
the MANTID data reduction framework, which is developed and maintained by an international
collaboration of neutron scattering user facilities for multiple instrument techniques includes
SANS, diffraction, and spectroscopy. As part of the development for drt-sans, several MANTID
algorithms were updated. In the final steps of data reduction, the lack of a dedicated structure to
hold information about both momentum and resolution has proved limiting, so custom classes based on
the NumPy libraries were created. The custom data structures are mixed strategically with the MANTID
workspaces to maximize performance and code reuse from community tools, which are well tested and
accepted. Extensive in- and out-of-source documentation is provided for drt-sans. The documentation
is auto-generated from the source code using the Sphinx documentation engine and is hosted on
user-accessible servers. Each documentation page links to related documentation as well as to
related source code.

Drt-sans has multiple deployments for users at ORNL. Development, quality assurance testing, and
production versions are available with production versions including native compiled binaries that
can be executed directly on Linux, Docker containers for virtualized environments, and Conda
packages that can be used natively but that do not require a particular version of the Linux
operating system. These packages are deployed on workstations, the neutrons analysis cluster, and
the ORNL Neutrons Jupyter Python Notebook server, all of which are available to users. Docker
containers, which encapsulate programs so they can run virtually on systems with different and
possibly unsuitable software configurations, for drt-sans are used by both developers and users. The
compiled binaries are used on the analysis cluster and are called by the Shaman web UI when it
provisions reduction tasks. While providing many different releases may seem confusing, it has its
purpose. First, having a diverse set of production releases is necessary to cover all common use
cases. Jupyter notebooks, for example, are an excellent fit for exploratory and educational uses of
the software compared with the native builds, which need to be optimized and deployed for constant
execution. Second, having development and testing builds facilitates the change management process
without interfering with stable production builds that are relied on by users for reduction.

Data files are generated by drt-sans when the reduction scripts finish execution. The output
includes numerical data files in formats suitable for data analysis, static images, and files that
can be used to make interactive images with the Shaman web interface. Additionally, the output
includes the scattering intensity for all detectors in both scalar (one-dimensional) and azimuthal
(two-dimensional) representations. 

Future of the drt-sans platform 

In many ways, the drt-sans platform represents a more compact and modular way of working with
algorithms already present in the MANTID framework and other libraries. There is significant market
pressure for software to become smaller, more flexible, and more service oriented, which is
different from years past in which large frameworks dominated ecosystems. 

Some parts of the drt-sans code can be reused for instruments that are outside of the SANS
instrument suite with only a very minor and reasonable amount of refactoring. However, much of the
drt-sans code base is highly specific to the SANS instrument technique. Core physical (wavelength)
and mathematical algorithms (binning) will be easy to reuse. These reusable parts, as well as the
general architecture of drt-sans, will form the basis and pattern for other packages in what will
become a larger data reduction toolkit that reaches across instrument suites. 

Examples

This section presents two examples of data that have been reduced using Shaman and drt-sans for the
instruments at SNS and HFIR. 

Silver Behenate (Ag-Be) on Bio-SANS and GP-SANS

As part of commissioning exercises when HFIR restarted in October 2019, the Bio-SANS and GP-SANS
teams performed experiments using Ag-Be. Ag-Be has 13 well-characterized diffraction peaks and can
be used in a variety of instruments to measure their performance during calibration. The following
image shows the two-dimensional scattering intensity of Ag-Be in the Bio-SANS main detector as
viewed in Shaman.

<INSERT-IMAGE:BIOSANS-IQXY>

The two-dimensional scattering intensity for the Bio-SANS wing detector can also be studied, as
shown in the following image.

<INSERT-IMAGE:BIOSANS-WING-DETECTOR>

Porous Silica Standard (Porasil B) on EQ-SANS

The following image shows a comparison between the data for a form of porous silica (porasil B)
using the old (pure MANTID) and new (drt-sans) data reduction codes. Porasil B is used for
calibration on EQ-SANS because of its stability and high scattering cross-section.

<INSERT-IMAGE: PORASIL>

Acknowledgments

The authors gratefully acknowledge the contributions of the following people in the preparation of
the software and this report. From the RSE and SCSE groups, this includes Jean Bilheux, Jose
Borreguero-Calvo, Caleb Cooper, Mathieu Doucet, Lance Drane, Ricardo Ferraz-Leal, Steve Hahn, Fahima
Islam, Jiao Lin, Vickie Lynch, Mark Martin, Marshall McDonnell, Alex Moore, Joseph Osborn, Peter
Peterson, Andrei Savici, Robert Smith, Ross Whitfield, and Wenduo Zhou. From the SANS Instrument
Team, this includes Wei-Ren Chen, Lisa Debeer-Schmitt, Changwoo Do, Michael Fitzsimmons, Lilin He,
William Heller, Ken Littrell, Sai Venkatesh Pingali, Shuo Qian, and Volker Urban, as well as former
members Matthew Cuneo, Christopher Stanley, and Bin Wu.

This work used resources of HFIR and SNS, which are US Department of Energy Office of Science User
Facilities operated by Oak Ridge National Laboratory under contract DE-AC05-00OR22725.




